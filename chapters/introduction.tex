\chapter{Introduction} \label{chap:introduction}

This chapter motivates the topic, frames the problem, states the objectives, and outlines the structure of the thesis.

\section{Motivation}
Mobile robotic systems are becoming increasingly integral to a diverse range of fields, including industry, research, and education. The development of applications for these complex systems fundamentally relies on robust processes for implementation, testing, and evaluation. A significant challenge in this process is bridging the divide between the controlled, flexible environment of pure software simulation and the high-fidelity, but often costly and hazardous, environment of real-world physical testing.

While initial validation can be effectively performed in simulated environments to gain early insights into system behavior, these virtual tests often fail to capture the full complexity of real-world dynamics. Consequently, physical field testing with actual hardware is crucial to uncover challenges and performance issues that are not apparent in a purely virtual setting. This creates a "testing gap" between the two methodologies. Pure simulation lacks physical realism, while physical testing is expensive, potentially dangerous, and difficult to conduct in a repeatable manner.

To bridge this gap, modern robotics development has embraced hybrid testing paradigms. The "X-in-the-Loop" (XitL) approach, particularly Robot-in-the-Loop (RitL) simulation, offers a powerful solution by integrating a physical robot with a simulated virtual environment. This method allows for the safe and repeatable testing of high-level algorithms on real hardware. The core technology enabling these frameworks is the Digital Twin, a virtualized, synchronized counterpart of a physical system that facilitates real-time monitoring and dynamic interaction. Furthermore, Mixed Reality (MR) technologies, including Augmented Reality (AR) and Virtual Reality (VR), are increasingly used to enhance human-robot interaction by providing intuitive interfaces for visualizing data and controlling the robot within these hybrid environments. This thesis is motivated by the need to create an advanced, high-fidelity platform that integrates these concepts to provide a more effective and scalable solution for the testing and validation of mobile robotic applications.

\section{Structure of the Thesis}
An overview of the following chapters.
