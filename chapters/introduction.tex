\chapter{Introduction} \label{chap:introduction}

Robotic and Autonomous Systems (RAS) combine multiple disciplines, including control engineering and robotics, mechanical engineering, electronics, and software engineering. While traditional testing relies on domain-agnostic software metrics like code coverage and failure counts, the multi-multidisciplinarity of RAS requires a extension of these methods to verify the system's autonomous operation, continuous movement, and ability to handle uncertainty. For researchers and engineers specializing in software testing, adapting existing techniques for RAS presents a significant challenge. This is why there is extensive literature dedicated to proposing and evaluating different testing techniques and processes, showing how essential it is to establish structured methods for ensuring the reliability of these complex systems. \cite{AMV23}

This validation challenge highlights a fundamental tension between simulation-only testing and full-scale physical experimentation, which are the two established methods in robotics evaluation. Simulation is central to the robotic development, offering a way to test control logic and experiment with system configurations before committing them to hardware \cite{Hu05, Mic04}. Simulations provide a rapid prototyping environment that minimizes infrastructure costs and logistical difficulties \cite{Mic04,Dos17}. Furthermore, they utilize virtual time to execute computationally expensive algorithms significantly faster than real-time hardware, enabling efficient design exploration \cite{Mic04}. A discontinuity often occurs during the transition to real hardware, where the models validated in simulation are discarded and the code is rewritten from scratch, introducing potential errors \cite{Hu05}. Additionally, simulation environments rarely replicate the full complexity of the physical world, missing nuances such as sensor noise, surface friction, and the unique manufacturing variances found in individual motors \cite{BM18}. Testing with real hardware allows for the highest fidelity and controls to be refined based on the presence of all the unpredictable physics and variability that simulations can't replicate \cite{Hu05, Cas21}. But this approach has its own drawbacks. Conducting physical experiments can be resource-intensive and time-consuming and require significant funds, manpower, and infrastructure \cite{Hu05, Dos17, Mic04}. Moreover, virtual environments are essential for validating scenarios that are hazardous or physically inaccessible in the real world, such as bomb disposal missions, space exploration, or pilot emergency training~\cite{BM18, Hu05}. Testing the entire system using only real components is also severely limited for complex, large cooperative systems because of issues relating to complexity and scale \cite{Hu05}. As neither of these approaches seems satisfactory on its own, an intermediate solution is required that could connect simulation with real-world experimentation \cite{Hu05}.

To bridge this gap, hybrid methods like "robot-in-the-loop" (RitL) simulation have emerged, which allow physical robots to operate within virtual environments \cite{Hu05}. This often centered on a Digital Twin, which is a real-time virtual replica of the physical robot \cite{AA23} and employs Augmented Reality (AR) as a Mixed Reality medium to facilitate user interaction~\cite{MV20}. The combination of these technologies provides a robust paradigm for the comprehensive testing and validation of robotic systems.

The “Virtual Environment for mobile Robotic Applications” (VERA) framework integrates digital twins, AR, and vehicle-in-the-loop testing together into a single system \cite{Geh24}. VERA provides a modular platform for creating, managing, and visualizing synchronized virtual environments, which are presented both in simulations and as real-world projections \cite{Geh24}. This master's thesis builds directly upon the foundation laid by this original framework.

While the VERA framework provides a strong conceptual foundation, its original technical implementation contained critical limitations regarding physical realism, scalability, and user interaction \cite{Geh24}. Its custom environment manager lacked a realistic physics engine, while its 2D visualizer, built with Pygame, showed performance degradation when handling a large number of dynamic objects, leading to delayed and incomplete visualizations \cite{Geh24}. Furthermore, while an AR interface was implemented, the original thesis identified immersive Virtual Reality (VR) interaction as a key area for future work \cite{Geh24}.

This thesis overcomes those limitations by offering a novel, hardware-accelerated architecture capable of real-time physics simulation replacing the custom manager and visualizer used within VERA. To address these needs, a game engine is selected to replace the custom solutions, specifically chosen to fulfill the requirements for high-fidelity graphics, integrated physics, and native support for VR. All communication and data synchronization between the physical robot and the simulation environment is carried out with the \textbf{Robot Operating System 2 (ROS~2)} \cite{MFG22}, a middleware framework used within state-of-the-art robotics.

The core goal is to create a real-time digital twin \cite{AA23} of the EMAROs robot \cite{Geh24}, integrating its complete model, live sensor data, and physical properties such as mass, inertia, and collision models. This digital twin serves as the foundation for robot-in-the-loop \cite{Hu05} testing scenarios. Additionally, this project extends the original Framework \cite{Geh24} to include Virtual Reality (VR) \cite{EM21} and desktop interfaces for scenario modification, robot control, and data visualization. This also includes reimplementing and enhancing VERA's AR floor projection feature by using the selected engine's rendering tools to improve visual quality and system capabilities.

\cite{NOT FINAL}The remainder of this thesis is structured as follows: after reviewing the basic concepts and technologies, such as RitL, Digital Twins, Mixed Reality, ROS~2, and several simulation engines in Chapter 2, an in-depth review of the original framework of VERA is presented, along with its limitations. In Chapter 3, the functional and non-functional requirements for the new system are identified. The architecture and implementation of the new framework are explained in detail in Chapter 4, with particular reference to the integration of **the game engine** with ROS~2, the development of the digital twin, and the mixed reality interaction system. Then, a quantitative evaluation regarding the performance of the system is presented along with a discussion of the results in Chapter 5. Finally, Chapter 6 summarizes the thesis contributions and gives an outlook on possible future research.\cite{NOT FINAL}