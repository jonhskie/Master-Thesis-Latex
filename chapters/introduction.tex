\chapter{Introduction} \label{chap:introduction}

Robotic and Autonomous Systems (RAS) combine multiple disciplines, including control engineering and robotics, mechanical engineering, electronics, and software engineering.  Testing these systems is not as straightforward as using traditional methods and they require more because they span multiple disciplines. For researchers and engineers specializing in software testing, adapting existing techniques for RAS presents a significant challenge. This is why there is extensive literature dedicated to proposing and evaluating different testing techniques and processes, showing how esential it is to establish structured methods for ensuring the reliability of these complex systems. \cite{AMV23}

This validation challenge highlights a fundamental tension between simulation-only testing and full-scale physical experimentation, which are the two established methods in robotics evaluation. Simulation is central to the robotic development, offering a way to test control logic and experiment with system configurations before committing them to hardware \cite{Hu05, Mic04}. Simulations are often easier to set up, less expensive, and can run faster than real-world tests, allowing for rapid design exploration and to get better grasp of complex algorithms \cite{Mic04, Dos17, BM18}. However, a gap between the virtual and physical worlds emerges when simulation models are abandoned during the implementation phase. Virtual models, by their nature, cannot perfectly represent every characteristic of a physical hardware, such as the exact effects of friction, sensor noise, or small differences in motors \cite{Hu05, BM18}. On the other hand, testing with real hardware provides the highest fidelity and allows for controls to be refined based on presence of all the unpredictable physics and variability, that simulations can’t replicate \cite{Hu05, Cas21}. But this approach has its own severe drawbacks. Conducting physical experiments can be resource-intensive and time-consuming and requires significant funds, manpower, and infrastructure \cite{Hu05, Dos17, Mic04}. Furthermore, some critical scenarios, such as emergencies, are too dangerous to be recreated in the physical world \cite{Hu05}. For complex, large cooperative systems, testing the entire system with only real components may not even be feasible due to the issues of complexity and scale \cite{Hu05}. Since neither approach on its own is sufficient, there is a need for an intermediate solution that can connect the simulation with real-world experimentation \cite{Hu05}.

To bridge this validation gap, hybrid paradigms such as Hardware-in-the-Loop-Simulation (HILS) have emerged, in which parts of a pure simulation are replaced with actual physical components \cite{Hu05}. This research focuses on a specific application of this concept known as "robot-in-the-loop" (RitL) simulation, a novel method that allows real robots and robot models to work together for system-wide measurement and test \cite{Hu05}. This paradigm provides the flexibility to experiment with real robots in a virtual environment, allowing them to be tested and evaluated even when a full physical environment is not available \cite{Hu05}. This hybrid interaction is enabled by a core enabling technology: the Digital Twin \cite{AA23}. A digital twin is a virtual copy of the physical thing that can monitor it in real time \cite{AA23}. Through connectivity and rapid processing, the digital twin is able to mirror the physical asset and synchronize data with it \cite{AA23}. To manage this complex hybrid test environment, Augmented Reality (AR) is increasingly utilized as a new medium for interaction and information exchange with autonomous systems, enhancing the efficiency of Human-Robot Interaction (HRI) \cite{MV20}. The powerful combination of RitL testing \cite{Hu05}, synchronized Digital Twins \cite{AA23}, and intuitive AR interfaces \cite{MV20} represents a state-of-the-art approach for developing and validating the next generation of complex robotic systems.

A system that integrates these concepts of digital twins, augmented reality (AR), and vehicle-in-the-loop testing is the "Virtual Environment for mobile Robotic Applications" (VERA) framework, which was designed to bridge the gap between simulation and real-world testing \cite{Geh24}. VERA provides a modular platform for creating, managing, and visualizing synchronized virtual environments, which are presented both in simulations and as real-world projections \cite{Geh24}. This master's thesis builds directly upon the foundation laid by this original framework.

While the VERA framework provided a promising conceptual foundation, its original implementation has several documented limitations that this thesis seeks to address \cite{Geh24}. A primary limitation was that its custom environment manager could not offer a full physics simulation, with the integration of enhanced physics noted as a direction for future work \cite{Geh24}. Furthermore, the 2D visualizer, based on Pygame, encountered performance degradation when handling a high number of dynamic objects; in scenarios with many updates, its capacity was exceeded, leading to delayed and incomplete visualizations \cite{Geh24}. Lastly, while AR-like projections were implemented, more immersive Virtual Reality (VR) interaction was not part of the original scope and was identified as a potential future extension \cite{Geh24}. Therefore, the central problem is that the original VERA framework, while conceptually promising, is technically constrained by its custom components, limiting its physical realism, performance scalability, and interaction capabilities.

To address these technical constraints, the primary objective of this thesis is to replace the custom simulation and visualization components of the VERA framework \cite{Geh24} with the Unity Engine \cite{Uni23}. This implementation will leverage the Unity Engine \cite{Uni23}, which was selected specifically for its advanced graphics, integrated physics, and native VR support, to overcome the previously identified constraints of the VERA system.

The core goal is to establish a high-fidelity, real-time digital twin \cite{AA23} of the EMAROS robot \cite{Geh24}, incorporating not only its model but also its live sensor data and physical properties, such as mass, inertia, and collision models. This digital twin will serve as the foundation for "robot-in-the-loop" \cite{Hu05} testing scenarios. Furthermore, this project will implement user interaction, moving beyond the original AR projections \cite{Geh24} to include Virtual Reality (VR) \cite{EM21} and desktop interfaces for scenario modification, robot control, and data visualization. This also includes reimplementing and enhancing VERA's AR floor projection feature by utilizing Unity's advanced rendering tools to improve visual quality and system capabilities \cite{Uni23}.

\cite{NOT FINAL}The remainder of this thesis is structured as follows: Chapter 2 reviews the foundational concepts of Robot-in-the-Loop testing, Digital Twins, and Mixed Reality, and includes a detailed review of the original VERA framework. Chapter 3 defines the new system requirements based on VERA's limitations. Chapter 4 details the implementation of the new architecture, including the Unity/ROS 2 integration, the EMAROS digital twin and the implemented scenarios. Chapter 5 presents the evaluation and discusses the results. Finally, Chapter 6 summarizes the thesis contributions and provides an outlook on future research.\cite{NOT FINAL}

