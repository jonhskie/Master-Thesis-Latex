\chapter{Introduction} \label{chap:introduction}

Robotic and Autonomous Systems (RAS) combine multiple disciplines, including control engineering and robotics, mechanical engineering, electronics, and software engineering. Testing these systems is not as straightforward as using traditional methods, and they require more because they span multiple disciplines. For researchers and engineers specializing in software testing, adapting existing techniques for RAS presents a significant challenge. This is why there is extensive literature dedicated to proposing and evaluating different testing techniques and processes, showing how essential it is to establish structured methods for ensuring the reliability of these complex systems. \cite{AMV23}

This validation challenge highlights a fundamental tension between simulation-only testing and full-scale physical experimentation, which are the two established methods in robotics evaluation. Simulation is central to the robotic development, offering a way to test control logic and experiment with system configurations before committing them to hardware \cite{Hu05, Mic04}. Simulations are often easier to set up, less expensive, and can run faster than real-world tests, allowing for rapid design exploration and to get a better grasp of complex algorithms \cite{Mic04, Dos17, BM18}. However, there is a gap between the virtual and physical worlds when simulation models are abandoned during the implementation phase. By nature, a virtual model cannot capture every characteristic of the physical hardware perfectly, such as the exact effects of friction, sensor noise, or small differences in motors \cite{Hu05, BM18}. Testing with real hardware allows for the highest fidelity and controls to be refined based on the presence of all the unpredictable physics and variability that simulations can't replicate \cite{Hu05, Cas21}. But this approach has its own severe drawbacks. Conducting physical experiments can be resource-intensive and time-consuming and require significant funds, manpower, and infrastructure \cite{Hu05, Dos17, Mic04}. Furthermore, some critical scenarios, such as emergencies, are too dangerous to be recreated in the physical world \cite{Hu05}. Testing the entire system using only real components is also not feasible for complex, large cooperative systems because of issues relating to complexity and scale \cite{Hu05}. As neither of these approaches seems satisfactory on its own, an intermediate solution is required that could connect simulation with real-world experimentation \cite{Hu05}.

To bridge this gap, hybrid methods like "robot-in-the-loop" (RitL) simulation have emerged, which allow physical robots to operate within virtual environments \cite{Hu05}. This is often centered on a "Digital Twin," which is a real-time virtual replica of the physical robot \cite{AA23}, and managed through Mixed Reality interfaces like Augmented Reality (AR) to enhance interaction \cite{MV20}. The combination of these technologies provides a robust paradigm for the comprehensive testing and validation of modern robotic systems.

The “Virtual Environment for mobile Robotic Applications” (VERA) framework integrates digital twins, AR, and vehicle-in-the-loop testing together into a single system \cite{Geh24}. VERA provides a modular platform for creating, managing, and visualizing synchronized virtual environments, which are presented both in simulations and as real-world projections \cite{Geh24}. This master's thesis builds directly upon the foundation laid by this original framework.

While the VERA framework provides a strong conceptual foundation, its original technical implementation contained various critical limitation regarding physical realism, scalability, and user interaction \cite{Geh24}. Its custom environment manager lacked a realistic physics engine, while its 2D visualizer, built with Pygame, showed performance degradation when handling a large number of dynamic objects, leading to delayed and incomplete visualizations \cite{Geh24}. Furthermore, while AR was implemented, the original thesis identified immersive Virtual Reality (VR) interaction as a key area for future work \cite{Geh24}.

This thesis overcomes those limitations by offering a modern, high-performance architecture replacing the custom manager and visualizer used within VERA. At the heart of this new implementation lies the \textbf{Unity Engine} \cite{Uni23}, a professional game engine selected due to its high-fidelity graphics, integrated physics, and native support for VR. All communications and data synchronization between the physical robot and the Unity simulation are carried out with the \textbf{Robot Operating System 2 (ROS~2)} \cite{MFG22}, the standard middleware framework used within modern robotics.

The core goal is to create a real-time digital twin \cite{AA23} of the EMAROS robot \cite{Geh24}, integrating its complete model, live sensor data, and physical properties such as mass, inertia, and collision models. This digital twin will serve as the foundation for "robot-in-the-loop" \cite{Hu05} testing scenarios. Additionally, this project will implement interaction by extending the original AR projections \cite{Geh24} to include Virtual Reality (VR) \cite{EM21} and desktop interfaces for scenario modification, robot control, and data visualization. This also includes reimplementing and enhancing VERA's AR floor projection feature by using Unity's rendering tools to improve visual quality and system capabilities \cite{Uni23}.

\cite{NOT FINAL}The remainder of this thesis is structured as follows: after reviewing the basic concepts and technologies, such as RitL, Digital Twins, Mixed Reality, ROS~2, and several simulation engines in Chapter 2, an in-depth review of the original framework of VERA is presented, along with its limitations. In Chapter 3, the functional and non-functional requirements for the new system are identified. The architecture and implementation of the new framework are explained in detail in Chapter 4, with particular reference to the integration of Unity with ROS~2, the development of the digital twin, and the mixed reality interaction system. Then, a quantitative evaluation regarding the performance of the system is presented along with a discussion of the results in Chapter 5. Finally, Chapter 6 summarizes the thesis contributions and gives an outlook on possible future research.\cite{NOT FINAL}

