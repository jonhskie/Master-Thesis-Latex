\chapter{Conclusion and Outlook}
\label{ch:conclusion_outlook}
\cite{NOT FINAL}
This thesis presented the design, implementation, and evaluation of a comprehensive Mixed Reality Environment tailored for Robot-in-the-Loop testing. The primary objective of this research was to overcome the inherent limitations of existing testing frameworks by developing a high-fidelity, scalable, and interactive platform. This platform successfully integrates physical robots into virtual simulations using digital twins, bridging the gap between pure simulation and real-world experimentation. By replacing the custom engine of the original VERA framework with a professional game engine and integrating it deeply with the Robot Operating System 2 (ROS 2), the new system provides a robust and modern architecture suited for the rigorous demands of robotic validation.

A central contribution of this work is the creation of a high-fidelity digital twin of the EMAROs robot. Unlike previous iterations, this digital replica is physically accurate and visually synchronized, capable of interacting with virtual physics in real time. This allows the physical robot to perceive and react to virtual obstacles as if they were real, enabling complex testing scenarios that would be difficult or costly to construct in the physical world. The integration of this digital twin ensures that the robot's logic can be validated against a sophisticated virtual environment while still operating within the physical laws of its real-world surroundings.

Furthermore, the system introduces immersive interfaces that significantly enhance the operator's ability to interact with the robotic simulation. The implementation of a Virtual Reality interface provides an intuitive platform for scenario modification and first-person supervision, placing the operator directly inside the testing environment. Complementing this is an enhanced Augmented Reality projection system that visualizes internal robot states and virtual map data directly onto the physical laboratory floor. These interfaces not only improve the transparency of the robot's decision-making process but also facilitate a seamless workflow for researchers designing and monitoring experiments.

The technical evaluation reveals significant performance improvements over previous frameworks. Comprehensive benchmarks confirm that the architecture allows for reliable operation at a steady 60 frames per second, even in environments containing up to 30,000 static or 3,000 dynamic objects. Crucially, the system maintains visualization latencies well below the threshold required for effective human-robot interaction, ensuring that the synchronization between the physical and virtual worlds remains imperceptible to the user. The flexibility of the system was further proven through the realization of four distinct application scenarios. These scenarios covered a broad range of essential robotic tasks, including dynamic surface interaction, reactive physics, autonomous logistics, and sensor-based decision-making in smart farming contexts. The successful execution of these diverse tasks demonstrates the platform's versatility and its readiness for broader adoption in robotics research.

Looking toward the future, the presented system establishes a stable foundation for Robot-in-the-Loop testing, yet several avenues for further research and improvement remain open. One primary area for optimization is the data transfer pipeline. The deep profiling analysis identified the transfer of raw image data between the GPU and CPU as a primary bottleneck in current operations. Future work could investigate the implementation of shared memory solutions or hardware-accelerated encoding to further reduce the latency of simulated camera sensors, thereby enabling even higher resolution inputs or faster update rates for high-speed robotic applications.

Another promising direction is the advancement of the physics simulation capabilities. Currently, the system utilizes rigid body physics to model interactions. Extending the simulation to support soft-body dynamics or fluid interactions would significantly broaden the scope of testable scenarios. This would enable the platform to be used for agricultural robotics where soil deformation is a factor, or for service robotics tasks involving the manipulation of deformable objects. Such enhancements would push the boundaries of what can be simulated in a mixed reality setting, bringing the virtual testing environment even closer to real-world complexity.

Beyond single-robot scenarios, the architecture holds potential for scaling towards Multi-Robot Systems. While the current focus is on a single digital twin, adapting the system to support Swarm-in-the-Loop testing is a logical next step. This would require optimizing network bandwidth management and state synchronization protocols to handle multiple digital twins simultaneously. Successfully addressing these challenges would allow researchers to study complex emergent behaviors and coordination strategies in large-scale robot swarms without the prohibitive cost of building dozens of physical prototypes.

Finally, while the technical performance of the VR and AR interfaces has been rigorously validated, the human element of the system warrants further investigation. Conducting qualitative user studies with robotics engineers and developers would provide valuable feedback on the usability, ergonomics, and workflow integration of the Mixed Reality tools. Understanding how operators interact with the system over prolonged periods could lead to interface refinements that further streamline the testing process. In summary, this thesis has established a powerful and modular framework for Mixed Reality robotics. Its open architecture ensures that it can evolve alongside the needs of the robotics community, facilitating the continued development of the next generation of autonomous systems.
\cite{NOT FINAL}
