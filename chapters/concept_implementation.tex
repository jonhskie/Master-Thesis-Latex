\chapter{Concept and Implementation}
\label{ch:implementation}

\cite{NOT FINAL}This chapter details the realization of the Mixed Reality Environment based on the requirements defined in Chapter 3. It begins with the selection of the simulation engine, followed by the system architecture, and concludes with the implementation of specific interaction logic and the autonomous ROS 2 agents.\cite{NOT FINAL}

\section{Technology Selection}
\label{sec:tech_selection}

To fulfill the platform capabilities defined in Section \ref{sec:platform_reqs} (Requirements FR-01 to FR-04), it is essential to select a suitable simulation engine. This section evaluates four industry-standard platforms: Gazebo, NVIDIA Isaac Sim, Unreal Engine, and Unity.

\subsection{Comparative Analysis}
The candidates were evaluated based on five key criteria: visual fidelity, physics capabilities, learning curve, community support, and native integration with XR and ROS 2. The results of this comparison are summarized in Table \ref{tab:sim_comparison}.

\begin{table}[H]
    \centering
    \caption{Comparison of Simulation Platforms for Mixed Reality Digital Twins~\cite{Gonzalez2025, Kargar2024, Singh2025, Coronado2023}.}
    \label{tab:sim_comparison}
    \begin{tabularx}{\textwidth}{@{}lXXXX@{}}
        \toprule
        \textbf{Feature} & \textbf{Gazebo} & \textbf{Isaac Sim} & \textbf{Unreal Engine} & \textbf{Unity} \\ 
        \midrule
        \textbf{Primary Use} & Control \& Navigation & AI \& Photorealism & Photorealism & HRI \& MR (VR/AR) \\ 
        \textbf{Visual Fidelity} & Moderate & Very High & Very High & High \\ 
        \textbf{Physics Engine} & ODE / Bullet & PhysX 5 (GPU) & Chaos / PhysX & PhysX \\ 
        \textbf{Learning Curve} & Steep & Advanced & Steep (C++) & Moderate (C\#) \\ 
        \textbf{Community} & High (ROS) & Moderate & High (Gaming) & Very High (MR) \\ 
        \textbf{ROS 2 Integ.} & Native & Bridge & Bridge & Plugin (Native) \\ 
        \textbf{Hardware} & Low & Very High (RTX) & High & Moderate \\ 
        \bottomrule
    \end{tabularx}
\end{table}

\subsection{Selection Rationale}
Based on the comparative analysis, the \textbf{Unity Engine} was selected as the implementation platform for this thesis. This decision is driven by four key factors that align with the system requirements:

\begin{itemize}
    \item \textbf{XR Framework (FR-03):} Unity offers a mature and integrated framework for Mixed Reality (AR/VR) applications~\cite{Coronado2023}. In contrast, traditional simulators like Gazebo lack native VR support, which is critical for the proposed human-robot interaction interface.
    
    \item \textbf{Visual \& Physics Balance (FR-01, FR-02):} Unity provides high-fidelity visualization alongside robust PhysX integration. This offers an optimal balance between performance and visual quality, avoiding the restrictive hardware requirements (e.g., RTX GPUs) associated with NVIDIA Isaac Sim~\cite{Gonzalez2025}.
    
    \item \textbf{Development Efficiency:} The use of C\# scripting, combined with extensive documentation and community support, makes Unity significantly more accessible for rapid prototyping than the C++ environment of Unreal Engine~\cite{Coronado2023}.

    \item \textbf{ROS 2 Integration (FR-04):} The availability of the \texttt{ros2-for-unity} asset enables the simulation to function as a first-party participant in the ROS 2 network. This satisfies the low-latency communication requirement by eliminating the need for external bridge applications~\cite{Rob24}.
\end{itemize}

% ==============================================================================
% 4.2 SYSTEM ARCHITECTURE
% ==============================================================================
\section{System Architecture}
\label{sec:system_architecture}

text
\subsection{High-Level Overview}

text
\subsection{Network Topology}
text
% ==============================================================================
% 4.3 UNITY AND ROS 2 INTEGRATION
% ==============================================================================
\section{Integration of Unity and ROS 2}
\label{sec:unity_integration}
text

\subsection{The ROS 2 for Unity Plugin}
text
\subsection{Simulation Time Management}
text
\subsection{Physical Twin Implementation}
text

\subsection{Sensor Simulation}
text
% ==============================================================================
% 4.4 MIXED REALITY INTERFACE
% ==============================================================================
\section{Mixed Reality Interface}
\label{sec:mr_interface}
text

\subsection{AR Projection}
text
\subsection{VR Interaction}
text
% ==============================================================================
% 4.5 SCENARIO LOGIC (UNITY-SIDE)
% ==============================================================================
\section{Implementation of Scenario Logic}
\label{sec:unity_scenarios}
text

\subsection{Smart Farming Environment}
text
\subsection{Logistics Environment}
text
\subsection{Surface Painting Mechanics}
text
\subsection{Adaptive Streets Scenario}
text
\subsection{Competitive Game and Unity Opponent}
text
% ==============================================================================
% 4.6 AUTONOMOUS AGENTS (ROS-SIDE)
% ==============================================================================
\section{Implementation of Autonomous Agents}
\label{sec:autonomous_agents}
text

\subsection{General Agent Architecture}
text
\subsection{Smart Farming Agent}
text
\subsection{Logistics Agent}
text
\subsection{Line Following Agent}
text
% ==============================================================================
% 4.7 SUMMARY
% ==============================================================================
\section{Summary of Implementation}
text